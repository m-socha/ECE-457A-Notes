\documentclass[12pt,titlepage]{article}
\usepackage[margin=1in]{geometry}

\begin{document}
  \begin{titlepage}
    \vspace*{\fill}
    \centering

    \textbf{\Huge ECE 457A Course Notes} \\ [0.4em]
    \textbf{\Large Cooperative and Adaptive Algorithms} \\ [1em]
    \textbf{\Large Michael Socha} \\ [1em]
    \textbf{\large 4A Software Engineering} \\
    \textbf{\large University of Waterloo} \\
    \textbf{\large Spring 2018} \\
    \vspace*{\fill}
  \end{titlepage}

  \newpage 

  \tableofcontents

  \newpage

  \section{Course Overview}
    \subsection{Logistics}
      \begin{itemize}
        \item \textbf{Professor:} Allaa Hilal
        \item \textbf{Email:} ahilal@uwaterloo.ca
        \item \textbf{Office:} EIT-3135
      \end{itemize}

  \section{Introduction}

    \subsection{What is Artificial Intelligence (AI)?}
      Intelligence is the ability to acquire and apply knowledge and skills. AI is the science of creating intelligent machines,
      including intelligent computer programs. Sample applications include visual perception, speech recognition and various
      forms of decision-making

      \subsubsection{AI vs Machine Learning vs Deep Learning vs Data Science}
        The above terms are often thrown around almost interchangealbly, but they refer to different things. Deep learning is
        a type of machine learning, which is a type of AI. Data science is a separate field that has some overlap with these
        three other concepts.

      \subsubsection{Inspiration from Nature}
        Some of the inspirations for intelligent systems come from the natural world. Sample inspirations include ant path-finding,
        bird flocking, and fish schooling.

    \subsection{Thinking Rationally vs Behaving Rationally}
      In rational thinking, logical systems are used to achieve goals via inferencing. It can be hard to represent informal knowledge,
      and not all problems can be solved through such methods (e.g. problems with uncertainty).

      In rational behavior, a perceives its environment and acts to achieve goals according to some set of beliefs. This is a more
      general approach than inferencing, and actions taken to achieve a goal may not necessarily be optimal or ``correct'', but
      may be an acceptable solution anyways.
      
    \subsection{Swarm Intelligence}
      Swarm intelligence is an AI technique that builds systems based on the collective behavior of decentralized, self-organized units.
      There are no centralized control structures, with agents only interacting with each other and the environment.

    \subsection{Intelligent Agents}
      An agent is something that senses its environment and acts on the collected information. A rational agent acts in a way that is
      expected to maximize performance on the basis of perceptual history and built-in knowledge. Types of agents include:
      \begin{itemize}
        \item \textbf{Simple Reflex Agents:} Follow a lookup-table approach; needs fully observable environment
        \item \textbf{Model-Based Reflex Agents:} Add state information to handle partially observable environments
        \item \textbf{Goal-Based Agents:} Add concept of goals to help choose actions
        \item \textbf{Utility-Based Agents:} Add utility to decide ``good'' or ``'bad'' when faced with conflicting goals
        \item \textbf{Learning Agents:} Add ability to learn from experience to improve performance
      \end{itemize}

    \subsection{Environments}
      The environments in which agents operate influence the design of the agents. Types of environments include:
      \begin{itemize}
        \item \textbf{Fully vs Partially Observable:} In fully observable environments, sensors can detect all aspects
        relevant to choice of action. This is not the case in partially observable environments, which may exist because
        of missing information or inaccurate sensors.
        \item \textbf{Deterministic vs Stochastic:} Deterministic environments are only influenced by their current
        state and the next action executed by the agent; otherwise, the environment is stochastic.
        \item \textbf{Episodic vs Sequential:} In episodic environments, the choice of action in each episode does not
        depend on previous episodes, unlike in sequential environments, where an agent needs to ``think ahead''.
        \item \textbf{Static vs Dynamic:} Environments where the state may change while the agent is deliberating are
        considered dynamic; other environments are considered static.
        \item \textbf{Discrete vs Continuous:} A discrete/continuous distinction can be applied to various aspects of
        an environment, including the way time flow is handled and to the actions of the agent.
        \item \textbf{Single vs Multi Agent}
      \end{itemize}

    \subsection{Adaptive and Cooperative Algorithms}
      Adaptive algorithms are able to change their behavior as they are run. Cooperative algorithms algorithms work
      together to solve a joint problem, communicating either directly or indirectly with one another.


\end{document}
